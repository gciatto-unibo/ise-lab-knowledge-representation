%%%%%%%%%%%%%%%%%%%%%%%%%%%%%%%%%%%%%%%%%%%%%%%%%%%%%%%%%%%%%%%%%%%%%%%%%%%%%%%%
% ISE Lab -- Knowledge Representation
% Giovanni Ciatto
% Alma Mater Studiorum - Università di Bologna
% mailto:giovanni.ciatto@unibo.it
%%%%%%%%%%%%%%%%%%%%%%%%%%%%%%%%%%%%%%%%%%%%%%%%%%%%%%%%%%%%%%%%%%%%%%%%%%%%%%%%
%\documentclass[handout]{beamer}\mode<handout>{\usetheme{default}}
%
\documentclass[presentation]{beamer}\mode<presentation>{\usetheme{AMSBolognaFC}}
%\documentclass[handout]{beamer}\mode<handout>{\usetheme{AMSBolognaFC}}
%%%%%%%%%%%%%%%%%%%%%%%%%%%%%%%%%%%%%%%%%%%%%%%%%%%%%%%%%%%%%%%%%%%%%%%%%%%%%%%%
\usepackage{ise-lab-common}
\usepackage{ise-lab-knowledge-representation}
% version
\newcommand{\versionmajor}{0}
\newcommand{\versionminor}{1}
\newcommand{\versionpatch}{4-dev}
\newcommand{\version}{\versionmajor.\versionminor.\versionpatch}
%%%%%%%%%%%%%%%%%%%%%%%%%%%%%%%%%%%%%%%%%%%%%%%%%%%%%%%%%%%%%%%%%%%%%%%%%%%%%%%%
\title[\currentLab{} -- Knowledge Representation]{Knowledge Representation with Horn Clauses}
%
\subtitle{\courseName{} / Module \moduleN{} (\courseAcronym)}
%
\author[\sspeaker{\gcShort}]{\speaker{\gcFull} \\ \gcEmail}
%
\institute[\disiShort, \uniboShort]{\disi{} (\disiShort)\\\unibo}
%
\date[A.Y. \academicYearShort{} (v.\ \version)]{Academic Year \academicYear{}\\(version \version)}
%
%%%%%%%%%%%%%%%%%%%%%%%%%%%%%%%%%%%%%%%%%%%%%%%%%%%%%%%%%%%%%%%%%%%%%%%%%%%%%%%%
\begin{document}
%%%%%%%%%%%%%%%%%%%%%%%%%%%%%%%%%%%%%%%%%%%%%%%%%%%%%%%%%%%%%%%%%%%%%%%%%%%%%%%%

%/////////
\frame{\titlepage}
%/////////

%%===============================================================================
\section*{Outline}
%%===============================================================================
%
%/////////
\frame[c]{\tableofcontents[hideallsubsections]}
%/////////

%===============================================================================
\section{Premises}
%===============================================================================

\begin{frame}{Lecture Goals}
    \begin{itemize}
        \item Understand basic notions concerning \alert{Horn clauses}
        %
        \begin{itemize}
            \item logic terms (constants, functions, variables)
            \item logic clauses (facts, rules, goals)
            \item unifiers and substitutions
            \item unification and MGU
        \end{itemize}

        \vfill

        \item Understand how these notions can be exploited for \alert{knowledge representation}
        %
        \begin{itemize}
            \item data structures representation in logic
            \item propositional vs. relational representations
            \item extensional vs. intensional representations
        \end{itemize}

        \vfill

        \item Understand differences in expressiveness among different logics
        %
        \begin{itemize}
            \item[eg] first-order logic vs. Horn clauses
        \end{itemize}
    \end{itemize}
\end{frame}

\begin{frame}{Motivations}
    \begin{itemize}
        \item Why Horn clauses?
        %
        \begin{itemize}
            \item nice expressiveness-tractability trade-off
            \item basis for Prolog, Datalog and Logic Programming
            \item very well established
            %
            \begin{itemize}
                \item involved in tons of literature, theorems, technologies
            \end{itemize}
        \end{itemize}

        \vfill

        \item Why (symbolic) knowledge representation?
        %
        \begin{itemize}
            \item pre-requisite for reifying many cognitive capabilities in software systems
            %
            \begin{itemize}
                \item[eg] reasoning, planning, deliberation, etc.
            \end{itemize}
            \item pre-requisite for logic programing
            \item both human- and machine-interpretable
        \end{itemize}
    \end{itemize}
\end{frame}

\begin{frame}[allowframebreaks]{Historical Overview}
    \begin{enumerate}
        \item In principle, it was first order logic (FOL)
        %
        \begin{itemize}
            \item very expressive, very flexible
            %
            \begin{itemize}
                \item[eg] (recursive) terms + variabiles + quantitiers + predicates + logic connectives
            \end{itemize}

            \item very hard to find a general \alert{resolution} algorithm for deciding the \alert{satisfiability} of any given FOL formula
        \end{itemize}

        \bigskip

        \item In 1965, Robinson proposes the \alert{SL resolution principle}
        %
        \begin{itemize}
            \item[ie] an algorithm for decising the \alert{unsatisfiability} of FOL formul\ae{} in \alert{Skolemized form}
            %
            \begin{itemize}
                \item Skolem form $\approx$ all variables are \alert{universally} quantified at the beginning of the formula
            \end{itemize}

            \item SL = Selective Linear
            \item logic \alert{unification} is a basic mechanism for the resolution principle
        \end{itemize}

        \framebreak

        \item In 1974, Kowalski proposes the SL\alert{D} resolution procedure
        %
        \begin{itemize}
            \item SL\alert{D} = Selective Linear [for] \alert{Definite} [clauses]
            %
            \begin{itemize}
                \item definite clauses $\approx$ Horn clauses
                \item[ie] very restricted subset of FOL, discussed in this lecture
                %
                \item[eg] no quantifiers, no negation, no connectors except conjunction and implication, \ldots
            \end{itemize}
        \end{itemize}

        \bigskip

        \item In the 1970s, first Prolog implementations appear
        %
        \begin{itemize}
            \item essentially reifying the SLD procedure into a programming language
            %
            \begin{itemize}
                \item cf. ``Fifty Years of Prolog and Beyond'' for the full history
            \end{itemize}
            \item that impacted virtually any subsequent logic/symbolic AI technology
        \end{itemize}

        \framebreak

        \item In 1978, Clark proposes extends SLD with \alert{negation as failure} (NaF)
        %
        \begin{itemize}
            \item adding well-founded negation support
        \end{itemize}

        \bigskip

        \item In 1982, Martelli and Montanari propose an efficient algorithm for unification
        %
        \begin{itemize}
            \item paving the way towards many sorts of automated reasoning algorithms / software
        \end{itemize}

        \bigskip

        \item[$\vdots$]

    \end{enumerate}
\end{frame}

%===============================================================================
\section{Knowledge Representation}
%===============================================================================

\begin{frame}{Overview}
    \begin{itemize}
        \item Three main ingredients:
        %
        \begin{description}\small
            \item[terms] --- for representing entities
            \item[predicates] --- for representing statements about entities
            \item[clauses] --- for representing properties of entities or relations among them
        \end{description}

        \vfill

        \item Many ways of representing knowledge through them:
        %
        \begin{description}\small
            \item[extensional vs. intensional] $\approx$ explicitly vs. implicitly
            \item[propositional vs. relational] $\approx$ in tabular form vs. as a graph
        \end{description}

        \vfill

        \item One powerful tool:
        %
        \begin{description}\small
            \item[resolution] --- allowing for \alert{intensional} representations, programming, reasoning, \ldots
        \end{description}

        \vfill

        \item Two fundamental mechanisms for manipulating knowledge:
        %
        \begin{description}\small
            \item[substitution application] $\approx$ rewriting a formula by assigning variables
            \item[most general unifier] $\approx$ computing the substitution making 2 formul\ae{} equal
        \end{description}
    \end{itemize}
\end{frame}

\subsection{Terms}

\begin{frame}[allowframebreaks]{Terms}
    \begin{block}{Purpose}\centering
        Terms are symbols representing entities from the \alert{domain of the discourse}
    \end{block}
    %
    \begin{block}{Informal definition}
        Terms can be
        %
        \begin{description}
            \item[constants] --- denoting \alert{individual} / simple entities
            \item[structures\footnote{a.k.a. functions}] --- denoting \alert{composed} / groups of entities
            \item[variables] --- denoting \alert{placeholders} for / reference of \emph{unknown} entities
        \end{description}
    \end{block}
    %
    \begin{alertblock}{Formal syntax\hfill\textbf{\footnotesize(notice the syntactic convention!)}}\label{slide:terms}
        \begin{center}
            $\begin{array}{rcl}
                \meta{Term} & := & \meta{Variable} \mid \meta{Structure} \mid \meta{Constant}
                \\
                \meta{Variable} & := & \variable{X}_1 \mid \variable{X}_2 \mid \variable{X}_3 \mid \ldots
                \\
                \meta{Structure} & := & \meta{Functor} \terminal{(} \meta{Arguments} \terminal{)}
                \\
                \meta{Functor} & := & \functor{f}_1 \mid \functor{f}_2 \mid \functor{f}_3 \mid \ldots
                \\
                \meta{Arguments} & := & \meta{Term} \mid \meta{Term} \terminal{,} \meta{Arguments}
                \\
                \meta{Constant} & := & \meta{Functor} \mid \meta{Number}\mid \meta{Boolean}
                \\
                \meta{Number} & := & \mathbb{R} \qquad \meta{Boolean} :=  \functor{true} \mid \functor{false}
            \end{array}$
        \end{center}
        %
        \begin{itemize}
            \item $\mathcal{X} = \{ \variable{X}_1, \variable{X}_2, \variable{X}_3, \ldots \}$ is a set of \alert{variables names}
            \item $\mathcal{F} = \{ \functor{f}_1, \functor{f}_2, \functor{f}_3, \ldots \}$ is a set of \alert{functors}\footnote{a.k.a. function names/symbols} of given \alert{arities}
            \item $\mathbb{R}$ is the set of real numbers
        \end{itemize}
    \end{alertblock}

    \begin{block}{Syntactical convention}
        \begin{itemize}
            \item $\variable{Variables}$ $\rightarrow$ capitalised italics
            \item $\functor{Functor}$ $\rightarrow$ lowercase monospaced
            \item $\meta{Non\text{-}terminal\ symbols}$ $\rightarrow$ sans-serif, wrapped by angular parenteses
            %
            \begin{itemize}
                \item this is just for grammar definitions
            \end{itemize}
            \item $\mathcal{SYMBOLS\ SET}$ $\rightarrow$ uppercase calligraphic italics
            %
            \begin{itemize}
                \item this is just for theoretical definitions
            \end{itemize}
        \end{itemize}
    \end{block}
\end{frame}

\begin{frame}[allowframebreaks]{Example -- Peano Numbers}
    \begin{exampleblock}{Definition: unary representation of natural numbers, via terms}
        \begin{description}
            \item[$\functor{z}$] $\rightarrow$ zero
            \item[$\functor{s}(\variable{X})$] $\rightarrow$ the successor of some (unknown) number $\variable{X}$
            \item[$\functor{s}(\functor{z})$] $\rightarrow$ the successor zero (a.k.a. 1)
            \item[$\functor{s}(\functor{s}(\functor{z}))$] $\rightarrow$ the successor of the successor of zero (a.k.a. 2)
            \item[$\functor{s}(\functor{s}(\functor{s}(\functor{z})))$] $\rightarrow$ the successor of \ldots
        \end{description}
    \end{exampleblock}
    %
    \begin{exampleblock}{Notice that, in this case:}
        \begin{itemize}
            \item $\functor{z}$ is a constant
            \item $\variable{X}$ is a variable
            \item $\functor{s}(\variable{X})$, $\functor{s}(\functor{z})$, etc. are structures
            \item $\mathcal{F} = \{ \functor{s}, \functor{z} \}$ \hfill {\footnotesize(where $\functor{s}$ is a 1-ary functor, while $\functor{z}$ is 0-ary)}
            \item $\mathcal{V} = \{ \functor{X} \}$
        \end{itemize}
    \end{exampleblock}
\end{frame}

\begin{frame}[allowframebreaks]{Example -- Lists}
    \begin{exampleblock}{Definition: (single-)linked lists\hfill\textbf{\footnotesize(LISP nomenclature\cccite{enwiki:cons})}}
        \begin{description}
            \item[$\functor{nil}$] $\rightarrow$ empty list
            \item[$\functor{cons}(\variable{H}, \variable{T})$] $\rightarrow$ the list whose head is $\variable{H}$ and whose tail is $\variable{T}$
            \item[$\functor{cons}(1, \variable{T})$] $\rightarrow$ the list whose first element is $1$ (and whose tail is $\variable{T}$)
            \item[$\functor{cons}(1, \functor{nil})$] $\rightarrow$ the singleton list $[1]$
            \item[$\functor{cons}(1, \functor{cons}(2, \functor{nil}))$] $\rightarrow$ the list $[1,2]$
            \item[$\functor{cons}(1, \functor{cons}(2, \functor{cons}(3, \functor{nil})))$] $\rightarrow$ the list $[1,2,3]$
        \end{description}
    \end{exampleblock}
    %
    \begin{exampleblock}{Notice that, in this case:}
        \begin{itemize}
            \item $\functor{nil}$ is a constant
            \item $\variable{H}$ and $\variable{T}$ are variables
            \item $\functor{cons}(\variable{H}, \variable{T})$, etc. are structures
            \item $\mathcal{F} = \{ \functor{nil}, \functor{cons} \}$ \hfill {\footnotesize(where $\functor{cons}$ is a binary functor, while $\functor{nil}$ is 0-ary)}
            \item $\mathcal{V} = \{ \variable{H}, \variable{T}\}$
        \end{itemize}
    \end{exampleblock}
    %
    \begin{alertblock}{Fun fact: in Prolog\dots}
        \begin{itemize}
            \item the constant `\alert{$\functor{[]}$}' is used in place of `\alert{$\functor{nil}$}'
            \item the binary functor `\alert{$\functor{.}$}' is used in place of `\alert{$\functor{cons}$}'
            \item `\alert{\pl{[$\variable{H}$ | $\variable{T}$]}}' is syntactic sugar for `\alert{\pl{.($\variable{H}$, $\variable{T}$)}}'
            \item `\alert{\pl{[1, 2, 3]}}' is syntactic sugar for `\alert{\pl{.(1, .(2, .(3, [])))}}'
        \end{itemize}
    \end{alertblock}
\end{frame}

\subsubsection{Ancillary definitions}

\begin{frame}[allowframebreaks]{Groundness}
    \begin{block}{Informal definition}\centering
        A term is \alert{ground} iff it has (i.e. contains) no variable
    \end{block}

    \begin{alertblock}{Inductive definition}
        \begin{itemize}
            \item Any variable $X$ is \emph{not} ground
            \item Any constant $\functor{c}$ is ground
            \item Any $n$-ary structure $\functor{f}(t_1, \ldots, t_n)$ is ground iff \emph{all} terms $t_i$ are ground
        \end{itemize}
    \end{alertblock}

    \begin{exampleblock}{Some examples}
        \begin{description}
            \item[$\functor{z}$] and \alert{$\functor{nil}$} $\rightarrow$ ground
            \item[$\functor{s}(\variable{X})$] $\rightarrow$ non-ground
            \item[$\functor{s}(\functor{z})$] $\rightarrow$ ground
            \item[$\functor{s}(\functor{s}(\functor{z}))$] $\rightarrow$ ground
            \item[$\functor{s}(\functor{s}(\functor{s}(\variable{X})))$] $\rightarrow$ non-ground
            \item[$\functor{cons}(\variable{H}, \variable{T})$] $\rightarrow$ non-ground
            \item[$\functor{cons}(1, \variable{T})$] $\rightarrow$ non-ground
            \item[$\functor{cons}(1, \functor{nil})$] $\rightarrow$ ground
            \item[$\functor{cons}(1, \functor{cons}(2, \variable{X}))$] $\rightarrow$ non-ground
            \item[$\functor{cons}(1, \functor{cons}(\variable{X}, \functor{cons}(3, \functor{nil})))$] $\rightarrow$ non-ground
        \end{description}
    \end{exampleblock}
\end{frame}

\begin{frame}[allowframebreaks]{Herbrand Universe}
    \begin{block}{Informal definition}
        The set of all items which can be represented as terms
        %
        \begin{itemize}
            \item attained by applying all $n$-ary functors to all possible terms, recursively
        \end{itemize}
    \end{block}

    \begin{alertblock}{Inductive definition}
        Let $\mathcal{F}$ be the set of all $n$-ary functors s.t. $n \geq 0$ (including constants)
        %
        \begin{itemize}
            \item let $\mathcal{H}_0$ be the set of all constants in $\mathcal{F}$
            \item let $\mathcal{H}_1$ be $\mathcal{H}_0 \cup \{ f(t_1, \ldots, t_n) \mid \forall f \in \mathcal{F}, \forall t_1, \ldots, t_n \in \mathcal{H}_0 \}$
            \item $\vdots$
            \item let $\mathcal{H}_{i+1}$ be $\mathcal{H}_i \cup \{ f(t_1, \ldots, t_n) \mid \forall f \in \mathcal{F}, \forall t_1, \ldots, t_n \in \mathcal{H}_i \}$
            \item $\vdots$
            \item then $\mathcal{H}_\infty$ is the Herbrand universe (spawned by $\mathcal{F}$)
        \end{itemize}
    \end{alertblock}

    \framebreak

    \begin{exampleblock}{Example for Peano numbers ($\mathcal{F} = \{ \functor{z}/0, \functor{s}/1 \}$)}
        \begin{itemize}
            \item $\functor{z}$
            \item $\functor{s}(\functor{z})$
            \item $\functor{s}(\functor{s}(\functor{z}))$
            \item $\functor{s}(\functor{s}(\functor{s}(\functor{z})))$
            \item[$\vdots$]
        \end{itemize}
    \end{exampleblock}

    \begin{exampleblock}{Example for lists ($\mathcal{F} = \{ \functor{nil}/0, \functor{cons}/2 \}$)}
        \begin{itemize}
            \item $\functor{nil}$
            \item $\functor{cons}(\functor{nil}, \functor{nil})$
            \item $\functor{cons}(\functor{cons}(\functor{nil}, \functor{nil}), \functor{nil})$
            \item $\functor{cons}(\functor{nil}, \functor{cons}(\functor{nil}, \functor{nil}))$
            \item $\functor{cons}(\functor{cons}(\functor{nil}, \functor{nil}), \functor{cons}(\functor{nil}, \functor{nil}))$
            \item[$\vdots$]
        \end{itemize}
    \end{exampleblock}

    \begin{alertblock}{Important take away}
        As soon as $\mathcal{F}$ contains \emph{at least}
        %
        \begin{itemize}
            \item one constant
            \item and one $n$-ary functor s.t. $n>0$
        \end{itemize}
        %
        the Herbrand base becomes of \alert{infinite} cardinality
    \end{alertblock}

    \begin{block}{Herbrand of \textbf{infinite} cardinality: implications}
        \begin{itemize}
            \item infinitely many terms can be represented
            \item attempts to enumerate them all won't terminate
            \item attempts to store them all will saturate the space
        \end{itemize}
    \end{block}
\end{frame}

\begin{frame}
    \todo[inline]{Exercise on term construction in \twopkt{}}

    \todo[inline]{Exercise on the construction of the first $N$ of the Herbrand universe for a given set of functors}
\end{frame}

\subsection{Predicates}

\begin{frame}[allowframebreaks]{Predicates}
    \begin{block}{Purpose}\centering
        Asserting statements about entities from the \alert{domain of the discourse}
    \end{block}
    %
    \begin{block}{Informal definition}
        A \alert{statement} about $n\geq 0$ \alert{terms}, which may or may not hold true
    \end{block}
    %
    \begin{alertblock}{Formal syntax\hfill\textbf{\footnotesize(notice the syntactic convention!)}}
        $$\begin{array}{rcl}
            \meta{Predicate} & := & \top \mid \bot \mid \meta{Predication} \mid \meta{Predication} \terminal{(} \meta{Arguments} \terminal{)}
            \\
            \meta{Predication} & := & \predication{p}_1 \mid \predication{p}_2 \mid \predication{p}_3 \mid \ldots
            \\
            \meta{Arguments} & := & \meta{Term} \mid \meta{Term} \terminal{,} \meta{Arguments}
            \\
            \meta{Term} & := & \text{see slide \ref{slide:terms}}
        \end{array}$$
        %
        \begin{itemize}
            \item $\mathcal{P} = \{ \predication{p}_1, \predication{p}_2, \predication{p}_3, \ldots \}$ is a set of \alert{predications}\footnote{a.k.a. predicate names/symbols} of given \alert{arities}
            \item \alert{$\top$} denotes the predicate which \alert{always} holds true (a.k.a. \alert{tautology})
            \item \alert{$\bot$} denotes the predicate which \alert{never} holds true (a.k.a. \alert{contradiction})
        \end{itemize}
    \end{alertblock}
    %
    \begin{block}{Syntactical convention}
        \begin{itemize}
            \item $\predication{predication}$ $\rightarrow$ lowercase italics
        \end{itemize}
    \end{block}
\end{frame}

\begin{frame}{Example -- Peano numbers}
    \begin{block}{Assumptions}
        \begin{itemize}
            \item $\mathcal{P} = \{ \predication{nat}, \predication{succ} \}$ \hfill {\footnotesize(where $\predication{nat}$ is a 1-ary predication, while $\predication{succ}$ is binary)}
            \item $\predication{nat}$ states that a term denotes a \alert{natural} number
            \item $\predication{succ}$ states that a term is the \alert{natural} of another one
            %
            \begin{itemize}
                \item provided that they are both natural numbers
            \end{itemize}
        \end{itemize}
    \end{block}
    %
    \begin{exampleblock}{Some possible predicates}
        \begin{description}
            \item[$\predication{nat}(\functor{z})$] $\rightarrow$ zero is a natural number
            \item[$\predication{nat}(\functor{s}(\variable{X}))$] $\rightarrow$ given some term $X$, the term $\functor{s}(\variable{X})$ is a natural number
            \item[$\predication{nat}(\functor{s}(\functor{z}))$] $\rightarrow$ one is a natural number
            \item[$\predication{succ}(\functor{s}(\functor{z}), \functor{z})$] $\rightarrow$ one is the successor of zero
            \item[$\predication{succ}(\functor{s}(\variable{X}), \variable{X})$] $\rightarrow$ given some term $X$, the term $\functor{s}(\variable{X})$ its successor
        \end{description}
    \end{exampleblock}
\end{frame}

\begin{frame}{Example -- Lists}
    \begin{block}{Assumptions}
        \begin{itemize}
            \item $\mathcal{P} = \{ \predication{list}, \predication{head} \}$ \hfill {\footnotesize(where $\predication{list}$ is a 1-ary predication, while $\predication{head}$ is binary)}
            \item $\predication{list}$ states that a term denotes a \alert{list}
            \item $\predication{head}$ states that a term is the \alert{head} of another one
            %
            \begin{itemize}
                \item provided that the latter is a list
            \end{itemize}
        \end{itemize}
    \end{block}
    %
    \begin{exampleblock}{Some possible predicates}
        \begin{description}
            \item[$\predication{list}(\functor{nil})$] $\rightarrow$ the empty list is a list
            \item[$\predication{list}(\functor{cons}(\variable{H}, \functor{nil}))$] $\rightarrow$ given some term $H$, the term $\functor{cons}(\variable{H}, \functor{nil})$ is a list
            \item[$\predication{list}(\functor{cons}(\variable{H}, \variable{T}))$] $\rightarrow$ given some terms $H, T$, the term $\functor{cons}(\variable{H}, \variable{T})$ is a list
            \item[$\predication{head}(\variable{H}, \functor{cons}(\variable{H}, \functor{nil}))$] $\rightarrow$ $H$ is the head of the list $\functor{cons}(\variable{H}, \functor{nil})$
            \item[$\predication{head}(\variable{H}, \functor{cons}(\variable{H}, \variable{T}))$] $\rightarrow$ $H$ is the head of the list $\functor{cons}(\variable{H}, \variable{T})$
        \end{description}
    \end{exampleblock}
\end{frame}

\begin{frame}{Important Remark}
    \begin{alertblock}{Predicates vs. structures}
        \begin{itemize}
            \item predicate and structures have a \emph{very} similar \alert{syntax}
            \item however, they are deeply different, \alert{semantically}
            %
            \begin{description}
                \item[terms] represent entities from the domain of the discourse
                %
                \begin{itemize}
                    \item they just exist
                \end{itemize}

                \item[predicates] represent statements about those entities
                %
                \begin{itemize}
                    \item they can either be true or false
                \end{itemize}
            \end{description}
        \end{itemize}
    \end{alertblock}
\end{frame}

\begin{frame}{Predicates and their arities}

    \begin{description}
        \item[0-ary predicate] (a.k.a. \alert{proposition}): denotes a statement which may either hold or not
        %
        \begin{itemize}
            \item e.g. ``it's raining''
        \end{itemize}

        \vfill

        \item[1-ary predicate] (a.k.a. \alert{set} or \alert{type}): denotes a group of items characterised by a given property
        %
        \begin{itemize}
            \item e.g. ``the set of even numbers''
            %
            \begin{itemize}
                \item[ie] ``the set of all numbers $X$ such that $X$ is a multiple of 2''
            \end{itemize}
        \end{itemize}

        \vfill

        \item[$n$-ary predicate] (a.k.a. \alert{relation}): denotes a relationship holding among $n$ entities
        %
        \begin{itemize}
            \item i.e. a group of $n$-uples characterised by a given property

            \item e.g. ``parenthood'' (binary relation)
            %
            \begin{itemize}
                \item[ie] ``the set of all pairs $(X, Y)$ such that $Y$ is a child of $X$''
            \end{itemize}

            \item e.g. ``students' yearly school reports'' (ternary relation)
            %
            \begin{itemize}
                \item[ie] ``the set of all triplets $(X, Y, Z)$ such that $X$ is a student, $Y$ is a course, and $Z$ is the mark of $X$ in $Y$''
            \end{itemize}
        \end{itemize}
    \end{description}
\end{frame}

\subsection{Horn Clauses}

\begin{frame}[allowframebreaks]{Horn Clauses}
    \begin{block}{Purpose}\centering
        Defining (a.k.a. expressing, writing) \alert{propositions}, \alert{sets}, or \alert{relations} concerning the entities of the \alert{domain of the discourse}
    \end{block}
    %
    \begin{block}{Informal definition}
        Horn clauses are logic formul\ae{} of three sorts:
        %
        \begin{description}
            \item[facts] --- denoting predicates which are known to hold
            \item[rules\footnote{a.k.a. definite clauses}] --- denoting that a predicate holds true if a number of other predicates hold true
            \item[goals\footnote{a.k.a. directives}] --- denoting a number of predicates to be proven (either true or false)
        \end{description}
    \end{block}
    %
    \begin{block}{Formal definition (pt. 1)}
        Horn clauses are logic statements of the form
        %
        \begin{center}
            $\overbrace{\underbrace{\phi}_{\text{positive literal}} \vee \underbrace{\neg\psi_1 \vee \ldots \vee \neg \psi_n}_{\text{negative literals}}}^{\text{disjunction form}}$
            \qquad $\equiv$\footnote{$(\neg a \vee b) \equiv (a \Rightarrow b)$, cf. \uuurl{http://discrete.openmathbooks.org/dmoi3/sec_propositional.html}} \qquad
            $\overbrace{\underbrace{\phi}_{\text{head}} \Leftarrow \underbrace{\psi_1 \wedge \ldots \wedge \psi_n}_{\text{body}}}^\text{implication form}$
        \end{center}
        %
        \begin{itemize}
            \item[ie] a \alert{disjunction} of \alert{literals} where \emph{at most} one literal is non-negated
            %
            \begin{itemize}
                \item[aka] an \alert{implication} having \emph{at most} 1 post-condition (the \alert{head}) and $n$ pre-conditions in conjunction (the \alert{the body})
            \end{itemize}
            \item where literals $\phi, \psi_1, \ldots, \psi_n$ are \alert{predicates} of any arity
            %
            \begin{itemize}
                \item possibly involving terms of any sorts
            \end{itemize}
        \end{itemize}
    \end{block}
    %
    \begin{block}{Formal definition (pt. 2)}
        Given a Horn clause $\phi \Leftarrow \psi_1 \wedge \ldots \wedge \psi_n$, it is
        %
        \begin{itemize}
            \item a \alert{goal} iff $\phi \equiv \bot$
            %
            \begin{itemize}
                \item[ie] when the head is a contradiction (a.k.a no head)
                \item then the clause is written as `$\Leftarrow \psi_1 \wedge \ldots \wedge \psi_n$'
            \end{itemize}

            \item a \alert{fact} if $n = 1$ and $\psi_1 = \top$
            %
            \begin{itemize}
                \item[ie] when the head is a tautology (a.k.a no body)
                \item then the clause is written as `$\phi$'
            \end{itemize}

            \item a \alert{rule} otherwise
        \end{itemize}
    \end{block}
\end{frame}

\begin{frame}{Examples -- Peano numbers}
    \begin{exampleblock}{Defining $\predication{nat}/1$ and $\predication{succ}/2$ via Horn clauses}
        \begin{itemize}
            \item TBD
        \end{itemize}
    \end{exampleblock}
\end{frame}

\begin{frame}{Examples -- Lists}
    \begin{exampleblock}{Defining $\predication{list}/1$ and $\predication{head}/2$ via Horn clauses}
        \begin{itemize}
            \item TBD
        \end{itemize}
    \end{exampleblock}
\end{frame}

\begin{frame}[allowframebreaks]{Horn clauses vs. FOL formul\ae}
    \begin{block}{Horn clauses are \textbf{particular cases} of FOL formul\ae, where}
        \begin{itemize}
            \item all variables in the head are \alert{universally} quantified
            \item all variables occurring in the in the body (but not in the head) are \alert{existentially} quantified
            \item where negation of predicates is forbidden
            \item where all logic connectives are forbidden
            %
            \begin{itemize}
                \item except conjunction and implication
                \item implication can only occur once
                \item conjunction can only occur among the pre-conditions of the implication
            \end{itemize}
        \end{itemize}
    \end{block}

    \begin{exampleblock}{How to read Horn clauses}
        \centering
        $\predication{mother(\variable{X}) \Leftarrow \predication{parent}(\variable{X}, \variable{Y}) \wedge \predication{female}(\variable{X})}$

        $\downarrow$ should be read as

        $\predication{\alert{\forall \variable{X}} : mother(\variable{X}) \Leftarrow \alert{\exists \variable{Y}} : \predication{parent}(\variable{X}, \variable{Y}) \wedge \predication{female}(\variable{X})}$
    \end{exampleblock}
\end{frame}

%===============================================================================
\section*{}
%===============================================================================

%/////////
\frame{\titlepage}
%/////////

%===============================================================================
\section*{\refname}
%===============================================================================

%%%%
\setbeamertemplate{page number in head/foot}{}
%/////////
\begin{frame}[c,noframenumbering]{\refname}
%\begin{frame}[t,allowframebreaks,noframenumbering]{\refname}
%	\tiny
    \scriptsize
%	\footnotesize
    \bibliographystyle{apalike-AMS}
    \bibliography{ise-lab-knowledge-representation}
\end{frame}
%/////////

%%%%%%%%%%%%%%%%%%%%%%%%%%%%%%%%%%%%%%%%%%%%%%%%%%%%%%%%%%%%%%%%%%%%%%%%%%%%%%%%
\end{document}
%%%%%%%%%%%%%%%%%%%%%%%%%%%%%%%%%%%%%%%%%%%%%%%%%%%%%%%%%%%%%%%%%%%%%%%%%%%%%%%%
